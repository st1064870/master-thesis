\section{Μαθηματικό Υπόβαθρο}
\label{section:math-backgorund}

\subsection{Αφηρημένη Άλγεβρα}

Σε αυτήν την ενότητα θα γίνει μια μικρή και ιδιαίτερα περιορισμένη εισαγωγή στην Αφηρημένη Άλγεβρα που είναι απαραίτητο εργαλείο στην μελέτη της κρυπτογραφίας. Σε καμία περίπτωση δεν είναι πλήρης και κυρίως θα μελετηθούν έννοιες που πιστεύουμε θα διευκολύνουν την ανάγνωση αποκλειστικά της συγκεκριμένης εργασίας. Ακόμα, παραλείπουμε τις αποδείξεις από κάθε θεώρημα, ωστόσο οι περισσότερες από αυτές είναι αρκετά απλές και μπορούν να βρεθούν σε ένα εισαγωγικό βιβλίο Αφηρημένης Άλγεβρας. Παροτρύνουμε τον μη μυημένο αναγνώστη που επιθυμεί να αποκτήσει μια πιο σφαιρική γνώση γύρω από το αντικείμενο αυτό να απευθυνθεί σε εισαγωγικά βιβλία, όπως το \cite{pinter2010book} το οποίο αποτελεί και κύρια βιβλιογραφική πηγή της παρούσας ενότητας.

Στην συνέχεια θα ξεκινήσουμε παραθέτοντας βασικούς ορισμούς της Αφηρημένης Άλγεβρας ξεκινόντας από τη θεωρία ομάδων και έπειτα συνεχίζοντας με τη θεωρία των δακτυλίων. Όπου κρίνεται απαραίτητο θα γίνονται σχόλια και αναφορές που έχουν άμεση σχέση με τα κρυπτογραφικά εργαλεία που χρησιμοποιούνται στα προηγούμενα κεφάλαια.

\begin{definition}
\textbf{Ομάδα (Group)} : Είναι η αλγεβρική δομή $\langle G, \circ \rangle$ που αποτελείται από ένα μη κενό σύνολο $G$, εφοδιασμένο μία πράξη $\circ : G \times G \rightarrow G$, που ικανοποιεί τις παρακάτω ιδιότητες:

\begin{itemize}
    \item Κλειστότητα : $\forall a,b \in G : a \circ b \in G$
    \item Προσεταιριστικότητα : $\forall a,b,c \in G : (a \circ b) \circ c = a \circ (b \circ c)$
    \item Ουδέτερο στοιχείο : $\exists! e \in G, \forall a \in G : a \circ e = a $
    \item Αντίστοφο στοιχείο : $\forall a \in G, \exists b \in G : a \circ b = e$
\end{itemize}
\end{definition}

Στα πλαίσια αυτής της ενότητας, μια ομάδα $\langle G, \circ \rangle$ την αποκαλούμε με το όνομα του συνόλου που εμπεριέχει, δηλαδή $G$ στην προκείμενη περίπτωση και όπου δεν κρίνεται αναγκαίο μπορεί να παραλείπεται η αναφορά στην πράξη που συνοδεύει την ομάδα $G$. Ακόμα η προκαθορισμένη σημειογραφία που θα χρησιμοποιηθεί για την πράξη της ομάδας είναι ο πολλαπλασιασμός (χωρίς αυτό να σημαίνει κάτι παραπάνω για την εσωτερική της δομή) και χρησιμοποιείται διαφορετικό σύμβολο μόνο όπου κρίνεται απαραίτητο για ευκολότερη ανάγνωση.

Ανάλογα με το είδος της πράξης, αν είναι για παράδειγμα πολλαπλασιασμός ή αφαίρεση, συνηθίζουμε να την αποκαλούμε \textbf{Πολλαπλασιαστική} ή \textbf{Προσθετική Ομάδα} αντίστοιχα και ουδέτερα στοιχεία τους 0 και 1 αντίστοιχα. Ωστόσο, αυτό δεν είναι καθοριστικό για την εσωτερική της δομή, καθώς πολλές ομάδες περιέχουν πράξεις που ουδεμία σχέση έχουν με τον πολλαπλασιασμό ή την αφαίρεση. Αν η ομάδα διαθέτει επίσης και την ιδιότητα της αντιμεταθετικότητας την αποκαλούμε \textbf{Αβελιανή Ομάδα}. Ακόμα, η \textbf{τάξη μια ομάδας} $G$, ορίζεται ως το πλήθος των στοιχείων της $ord(G)=|G|$, σε περίπτωση που είναι πεπερασμένη η τάξη της η ομάδα ονομάζεται \textbf{Πεπερασμένη Ομάδα}.

\begin{definition}
\textbf{Υποομάδα (Subgroup)} : Έστω μια ομάδα $\langle G, \circ \rangle$ και $\emptyset \subset Η \subseteq G$. Η αλγεβρική δομή $\langle H, \circ \rangle$ εφοδιασμένη με την ίδια ακριβώς πράξη $\circ$ ονομάζεται υποομάδα της $G$ και την συμβολίζουμε $H \leq G$ αν, και μόνο αν είναι κλειστή ως προς την $\circ$ και την ύπαρξη αντίστροφου. 
\end{definition}

Μπορούμε τώρα να ορίσουμε μια ειδική μορφή υποομάδας που θα αποδειχθεί χρήσιμη στη συνέχεια.

\begin{definition}
\textbf{Κανονική Υποομάδα} : Έστω οι ομάδες $G$, $H$ με $G \leq H$. Το $H$ είναι κανονική υποομάδα αν είναι κλειστή ως προς την συζυγία και την συμβολίζουμε $H \trianglelefteq G$. Δηλαδή :
$$
\forall a \in H, x \in G : xax^{-1} \in H
$$
\end{definition}

Μια ακόμη πολύ σημαντική κατηγορία ομάδων που έχουν μεγάλη χρησιμότητα στον κλάδο της κρυπτογραφίας είναι αυτή των κυκλικών ομάδων.

\begin{definition}
\textbf{Κυκλική Ομάδα (Cyclic group} : Μια ομάδα $G$ όπου κάθε στοιχείο της $a \in G$ είναι μπορεί να εκφραστεί ως δύναμη κάποιου συγκεκριμένου στοιχείου της $g \in G$, το οποίο ονομάζεται \textbf{γεννήτορας} (\textbf{generator}) της κυκλικής ομάδας. Δηλαδή :
$$
\exists g \in G, \forall a \in G, \exists n \in \mathbb{Z} : a = g^n
$$
\end{definition}

Στο σημείο αυτό μπορούμε να ορίσουμε την συνάρτηση του Διακριτού Λογαρίθμου, μια πολύ γνωστή συνάρτηση στην κρυπτογραφία.

\begin{definition}
\textbf{Διακριτός Λογαρίθμος (Discrete Logarithm ή DL)} : Έστω μια κυκλική ομάδα $\langle G, \cdot \rangle$, τότε γνωρίζουμε ότι $G = \langle g \rangle$ όπου $g$ ένας γεννήτορας της ομάδας, δηλαδή  $\forall a \in G,  \exists x \in |G|: g^x=a \in G$. Ο Διακριτός Λογάριθμος ορίζεται ως η συνάρτηση $log_g(a)$ που επιστρέφει την δύναμη $x$ για την οποία $g^x=a$, δηλαδή $log_g(a)=x \Leftrightarrow g^x=a$.
\end{definition}

Από τον Διακριτό Λογάριθμο προκύπτει και το γνωστό \textbf{Πρόβλημα του Διακριτού Λογαρίθμου (Discrete Logarithm Problem ή DLP)} το οποίο έγκειται στον υπολογισμό την συνάρτησης του DL με κάποιον γενικευμένο αλγόριθμο αναζήτησης. Το πρόβλημα αυτό θεωρείται πως στο κλασσικό μοντέλο υπολογισμού ότι είναι δυσεπίλυτο δηλαδή ότι δεν υπάρχει πολυωνυμικός στον αριθμό των ψηφίων της ομάδας $G$ (δηλαδή του μεγέθους της τάξης της ομάδας) αλγόριθμος που να το επιλύει.

Στην συνέχεια μπορούμε να δούμε πως ορίζονται οι βασικές ιδιότητες του ομομορφισμού και του ισομορφισμού οι οποίες απαντώνται στα Ομομορφικά Κρυπτογραφικά Σχήματα αλλά και έχουν σημαντικό ρόλο στο παρακάτω θεώρημα.

\begin{definition}
\textbf{Ομομορφισμός Ομάδας (Group Homomorphism)} : Έστω $\langle G, \circ \rangle$, $\langle H, \star \rangle$ ομάδες, μια συνάρτηση $f : G \rightarrow H$ είναι ομομορφισμός αν και μόνο αν :
$$
\forall a, b \in G : f(a \circ b) = f(a) \star f(b)
$$
\end{definition}

\begin{definition}
\textbf{Ισομορφισμός (Isomoprhism)} : Μια συνάρτηση $f$ είναι ισομορφισμός αν, και μόνο αν, είναι ομομορφισμός και αμφιμονοσήμαντη.
\end{definition}

Σε αυτό το σημείο μπορούμε να αναφερθούμε σε ένα θεμελιώδες θεώρημα της Θεωρίας Ομάδων, το οποίο βρίσκεται πίσω από πολλές ιδιότητες των ομάδων και των δακτυλίων που χρησιμοποιούνται στην κρυπτογραφία.

\begin{theorem}
\textbf{Θεώρημα Lagrange (Lagrange's theorem)} :  Σε κάθε πεπερασμένη ομάδα $G$, η τάξη κάθε υποομάδας της $Η \leq G$ διαιρεί την τάξη της $G$, δηλαδή 
$$
\forall H \leq G, ord(G)|ord(H)
$$
\end{theorem}

Από το παραπάνω μπορεί να αποδειχθεί το εξής θεώρημα :
\begin{theorem}
Κάθε πεπερασμένη ομάδα πρώτης τάξης είναι κυκλική και κάθε στοιχείο της είναι γεννήτορας της ομάδας.
\end{theorem}

Οι επόμενοι δύο ορισμοί είναι απαραίτητα εργαλεία στην κατανόηση του παρακάτω θεωρήματος.

\begin{definition}
\label{def:coset}
\textbf{Συσύνολο Ομάδας (Group Coset)} : Έστω $\langle G, \circ \rangle$ με $G = \{a_1, a_2, \ldots\}$ (η ομάδα μπορεί να είναι πεπερασμένη ή και μη) μια πολλαπλασιαστική ομάδα και μια υποομάδα $Η$ της $G$. Για κάθε στοιχείο $a \in G$ ορίζουμε τα παρακάτω :
\begin{itemize}
    \item Αριστερό Συσύνολο : $a \circ H = \{a \circ a_1, a \circ a_2, \ldots\}$
    \item Δεξί Συσύνολο : $H \circ a = \{a_1 \circ a, a_2 \circ a, \ldots\}$
\end{itemize}
\end{definition}

Για τα συσύνολα ομάδων αφού ως προκαθορισμένη σημειογραφία για την πράξη της ομάδας είναι ο πολλαπλασιασμός προκύπτουν οι συμβολισμοί $a Η$ και $H a$ αντίστοιχα.

\begin{definition}
\textbf{Ομάδα Υπολοίπου} : Έστω οι G, H του Ορισμού \ref{def:coset} με τον περιορισμό ότι $H \trianglelefteq H$. Ορίζουμε τον παρακάτω συμβολισμό $G / H = \{Ha_1, Ha_2, Ha_3\}$. Το σύνολο $G / H$ εφοδιασμένο με την πράξη του πολλαπλασιασμού συσυνόλων, μπορεί να αποδειχτεί ότι αποτελεί ομάδα και το ονομάζουμε \textbf{Ομάδα Υπολοίπου της G προς το H}.
\end{definition}

\begin{definition}
\textbf{Θεμελιώδες Ομομορφικό Θεώρημα (\EN{Fundamental Homomorphism Theorem (FHT)})} : Έστω $G$,$H$ δύο ομάδες όπου $H \trianglelefteq G$ και $f : G \rightarrow H$ ένας ομομορφισμός από το $G$ στο $H$. Αν $K$ είναι ο πυρήνας της $f$ τότε :
$$
Η \cong G / K
$$
\end{definition}

Στην συνέχεια θα γίνει μια μικρή εισαγωγή στη θεωρία των δακτυλίων, η οποία είναι ιδιαίτερα περιορισμένη και δίνεται έμφαση όπου αυτό κρίνεται απαραίτητο. Θα παρατηρήσουμε πως πολλές έννοιες είναι παρόμοιες με αυτές των ομάδων και για αυτό δεν παρέχονται παντού σχόλια.

\begin{definition}
\textbf{Δακτύλιος} : Είναι αλγεβρική δομή $\langle R, \circ, \star \rangle$, η οποία αποτελείται από ένα μη κενό σύνολο $R$, εφοδιασμένο με δύο διμελείς πράξεις $\circ:R \times R \rightarrow R$ και $\star: R \times R \rightarrow R$, που ικανοποιεί τα ακόλουθα αξιώματα :

\begin{itemize}
    \item Η αλγεβρική δομή $\langle R, \circ \rangle$ είναι αβελιανή ομάδα.
    \item Η αλγεβρική δομή $\langle R, \star \rangle$ είναι μονοειδές.
    \item Η πράξη $\star$ είναι επιμεριστική ως προς την $\circ$ :
    \begin{itemize}
        \item Αριστερός Επιμεριστικός νόμος : $\forall a, b, c \in R : a \star (b \circ c) = (a \star b) \circ (a \star c)$
        \item Δεξιός Επιμεριστικός νόμος :$\forall a, b, c \in R : (b \circ c) \star a = (b \star a) \circ (c \star a)$
    \end{itemize}
    \end{itemize}
\end{definition}

Αντίστοιχα με την περίπτωση των ομάδων έτσι και με αυτή των δακτυλίων θα ακολουθήσουμε παρόμοια τακτική στην σημειολογία. Με την μόνη διαφορά ότι στην περίπτωση των δακτυλίων η προκαθορισμένη σημειογραφία που θα χρησιμοποιήσουμε για τις πράξεις είναι αυτή της πρόσθεσης και του πολλαπλασιασμού για την πρώτη και την δεύτερη πράξη ενός δακτύλιο αντίστοιχα.

Σε αυτό το σημείο πρέπει να αναφέρουμε ότι σε ένα μικρότερο ποσοστό της βιβλιογραφίας ο παραπάνω ορισμός του δακτυλίου αναφέρεται ως δακτύλιος με μονάδα, αναφερόμενος στο πολλαπλασιαστικό ταυτοτικό στοιχείο. Στη περίπτωση που η δομή του δακτυλίου δεν εμπεριέχει πολλαπλασιαστικό ταυτοτικό στοιχείο τότε η δομή του πολλαπλασιασμού μετατρέπεται σε ημι-ομάδα (semi-group). Η σύμβαση που ακολουθείται ως επί το πλείστον στην βιβλιογραφία, είναι σύμφωνη με τον παραπάνω ορισμό του δακτυλίου που δώσαμε ενώ η δομή χωρίς την πολλαπλασιαστική μονάδα απαντάται ως \textbf{Δακτύλιος χωρίς μονάδα} (Ring without "i"dentity ή Rng). Ένας δακτύλιος ο οποίος είναι αντιμεταθετικός ως προς τον πολλαπλασιασμό αποκαλείται \textbf{Αντιμεταθετικός Δακτύλιος}.

\begin{definition}
\textbf{Υποδακτύλιος (Subring)} : Έστω ένας δακτύλιος $\langle R, \circ, \star \rangle$ και $\emptyset \subset Η \subseteq G$. Η αλγεβρική δομή $\langle H, \circ, \star \rangle$ εφοδιασμένη με τις ίδιες ακριβώς πράξεις με αυτή του $R$ είναι υποδακτύλιος αν και μόνο αν είναι κλειστή ως προς την $\circ$, τον αντίστροφο ως προς την $\circ$ και κλειστή ως προς την $\star$.
\end{definition}

\begin{definition}
\textbf{Ιδεώδες (Ideal)} : Έστω οι δακτύλιοι $A$, $B$ με το $B$ να είναι υποδακτύλιος του $Α$. Το $B$ είναι ιδεώδες αν, και μόνο αν απορροφά τον πολλαπλασιασμό στο $A$. Δηλαδή :
$$
\forall a \in A, \forall b \in B : ab \in B
$$
\end{definition}

\begin{definition}
\textbf{Ομομορφισμός Δακτυλίου (Ring Homomorphism)} : Έστω $\langle A, \circ_1, \star_1 \rangle$, $\langle B, \circ_2 \star_2 \rangle$ ομάδες, μια συνάρτηση $f : A \rightarrow B$ είναι ομομορφισμός αν και μόνο αν :
\begin{itemize}
    \item $\forall a, b \in A : f(a \circ_1 b) = f(a) \circ_2 f(b)$
    \item $\forall a, b \in A : f(a \circ_2 b) = f(a) \star_2 f(b)$
\end{itemize}
\end{definition}

\begin{definition}
\textbf{Συσύνολο Δακτυλίου (Ring coset)} : Έστω ένας δακτύλιος $\langle A, \circ, \star$ με $Α = \{a_1, a_2, \ldots \}$ και $B$ ένα ιδεώδες του Α. Για κάθε στοιχείο $a \in A$ το συσύνολο δακτυλίου ορίζεται ως εξής :
$$
a \circ J = \{a \circ a_1, a \circ a_2, \ldots\}
$$
\end{definition}

Για τα συσύνολα δακτυλίων αφού ως προκαθορισμένη σημειογραφία για την πρώτη πράξη είναι η πρόσθεση προκύπτει ο συμβολισμός $a + J$. Προφανώς στην περίπτωση των δακτυλίων αφού ως προς την πρόσθεση είναι αβελιανή ομάδα έχουμε $a + J = J + a$.

\begin{definition}
\textbf{Δακτύλιος Υπολοίπου (Quotient Ring)} : Έστω το $Α$ ένας δακτύλιος και το $J$ ένα ιδεώδες του. Το $A / J$ εφοδιασμένο με τις πράξεις της πρόσθεσης και του πολλαπλασιασμού συσυνόλων είναι δακτύλιος και ονομάζεται \textbf{Δακτύλιος Υπολοίπου του A προς το J}
\end{definition}

\begin{definition}
\textbf{Θεμελιώδες Ομομορφικό Θεώρημα για δακτύλιους (\EN{Fundamental Homomorphism Theorem (FHT)})} : Έστω δύο δακτύλιοι $A$ και $B$ όπου το $B$ είναι ιδεώδες του $A$ και $f : A \rightarrow B$ ένας ομομορφισμός από το $A$ στο $B$ με $K = ker(f)$. Τότε μπορεί να αποδειχθεί ότι :
$$
Β \cong A / K
$$
\end{definition}

Τέλος, ένα πολύ χρήσιμο εργαλείο στην κρυπτογραφία αλλά και γενικότερα στην υλοποίηση υπολογισμών, όπως θα δούμε και στο Κεφάλαιο \ref{chapter:SMPC} είναι μια ειδική κατηγορία δακτυλίων, τα πεδία.

\begin{definition}
\textbf{Πεδίο (Field)} : Ένας αντιμεταθετικός δακτύλιος στον οποίο κάθε στοιχείο έχει πολλαπλασιαστικό και προσθετικό αντίστροφο.
\end{definition}

\begin{definition}
\textbf{Πεπερασμένο πεδίο (Finite field ή Galois Field)} : Ένα πεδίο με πεπερασμένα στοιχεία.
\end{definition}

Μπορεί να αποδειχθεί ότι η τάξη ενός πεπερασμένου πεδίου είναι πάντα πρώτος αριθμός ή δύναμη πρώτου αριθμού. 

\subsection{Θεωρία αριθμών}

Στην αναφορά μας στη Θεωρία Αριθμών θα θα αρκεστούμε μόνο σε απαραίτητα συμπεράσματα τα οποία απαιτούνται για την κατανόηση της συγκεκριμένης εργασίας. Το παρακάτω θεώρημα είναι ένα από τα πολύ βασικά στην κρυπτογραφία και είναι έμμεση απορία του Θεωρήματος του Lagrange.

\begin{definition}
\textbf{Μικρό Θεώρημα του Fermat (Fermat's Little Theorem ή FLT)} :  Έστω μια ομάδα $\langle G, \circ \rangle$ με τάξη  $ord(G)=p$, όπου $p$ ένας πρώτος αριθμός, τότε μπορεί να αποδειχθεί ότι :
$$
\forall g \in G : g^{p} \equiv g (\pmod p)
$$
\end{definition}

Προφανώς, αφού κάθε ομάδα πρώτης τάξης είναι και κυκλική, μπορούμε να εξάγουμε και άλλες ταυτότητες από το παραπάνω θεώρημα, όπως ότι $\forall g \in G : g^{p-1} \equiv 1 (\pmod p)$. Στην συνέχεια θα αναφερθούμε σε μια πολύ σημαντική συνάρτηση για τον κλάδο της κρυπτογραφίας.

\begin{definition}
\textbf{Συνάρτηση Euler (Euler's totient function)} :  Ορίζεται ως η συνάρτηση $φ: \mathbf{Z} \rightarrow \mathbf{Z}$, $φ(n)$ και επιστρέφει τον αριθμό των σχετικά πρώτων αριθμών με τον αριθμό του ορίσματος. Μπορεί να υπολογιστεί από τον παρακάτω τύπο :
$$
φ(n)=n\prod_{p\mid n}\left(1-{\frac {1}{p}}\right)
$$
\end{definition}

Η συγκεκριμένη συνάρτηση έχει αποδειχθεί ότι για οποιονδήποτε μη πρώτο αριθμό $n$, ο υπολογισμός της είναι ισοδύναμος με το πρόβλημα της \textbf{Παραγοντοποίησης σε γινόμενο πρώτων παραγόντων} \cite{miller1976riemann}. Στην ισοδυναμία αυτή βασίζεται η ασφάλεια πολλών κρυπτογραφικών σχημάτων, όπως του γνωστού RSA. Το τελευταίο πρόβλημα εικάζεται ότι ανήκει στην κλάση προβλημάτων $co-NP \cap NP$ και μια από τις πολύ βασικές κρυπτογραφικές υποθέσεις είναι ότι το συγκεκριμένο πρόβλημα είναι δυσεπίλυτο στο κλασσικό μοντέλο υπολογισμού. Αντιθέτως, έχει αποδειχθεί ότι στο κβαντικό μοντέλο υπολογισμού ανήκει μπορεί να υπολογιστεί αποδοτικά από τον Αλγόριθμο του Shor και ότι ανήκει και στην κλάση $BQP$. 

\subsection{Άλγεβρα}

\subsubsection{Αναπαράσταση Πολυωνύμου}

Στην παρούσα Ενότητα παρουσιάζεται, χωρίς ιδιαίτερη εμβάθυνση, κυρίως το μαθηματικό υπόβαθρο που απαιτείται για να γίνει απολύτως κατανοητή η Διαμοίραση Μυστικών Shamir που μελετάμε στο Κεφάλαιο \ref{chapter:SMPC} αλλά και η περίπτωση της πράξης του πολλαπλασιασμού του Πρωτόκολλου BGW που μελετάμε επίσης στο ίδιο κεφάλαιο.

\begin{theorem}
\textbf{Ισοδυναμία Αναπαράστασης Πολυωνύμου} : Κάθε πολυώνυμο $n$-οστού βαθμού $α(x) = α_nx^n + α_{n-1}x^n + \ldots + α_0x + γ$ μπορεί να περιγραφεί κατά μοναδικό τρόπο είτε από τους $n+1$ συντελεστές του $[α_n, α_{n-1}, \ldots, α_0, γ]$ είτε από $n+1$ διαφορετικές αποτιμήσεις του πολυωνύμου $[α(x_1), α(x_2), \ldots, α(x_{n+1}]$ στα σημεία $x_1, x_2, \ldots, x_{n+1}$. Δηλαδή αυτές οι δύο αναπαραστάσεις είναι ισοδύναμες.
\end{theorem}

Η απόδειξη του παραπάνω είναι κατασκευαστική είναι πολύ απλή. Ας υποθέσουμε ότι $\mathbf{c} = [α_n, α_{n-1}, \ldots, α_0, γ]^T$ το διάνυσμα των συντελεστών του $α(x)$ και ότι $\mathbf{c} = [α(x_1), α(x_2),$ $\ldots, α(x_{n+1}]^T$ το διάνυσμα των αποτιμήσεων. Υπάρχει μια $1-1$ συνάρτηση η οποία μετατρέπει από την μια αναπαράσταση στην άλλη. H συνάρτηση για την μετατροπή από την περιγραφή μέσω συντελεστών του πολυωνύμου στην περιγραφή μέσω αποτιμήσεων είναι η παρακάτω :

$$
    \mathbf{V} = 
    \begin{bmatrix}
    1 & α_{1} & α_{1}^{2} & \ldots & α_{1}^{2 t} \\
    1 & α_{2} & α_{2}^{2} & \ldots & α_{2}^{2 t} \\
    \vdots & & & & \\
    1 & α_{n} & α_{n}^{2} & \ldots & α_{n}^{2 t}
    \end{bmatrix}
$$

και μπορεί να εφαρμοστεί ως εξής :

$$
     \mathbf{s} = \mathbf{V} \times \mathbf{c}
$$

Αντίστοιχα μπορούμε επειδή το $\mathbf{V}$ είναι μητρώο Vandermonde είναι και αντιστρέψιμο. Έτσι μπορούμε να υπολογίσουμε το διάνυσμα συντελεστών $\mathbf{c}$ από το διάνυσμα αποτιμήσεων $\mathbf{s}$ ως εξης :

$$
    \mathbf{c} = \mathbf{V}^{-1} \times \mathbf{s}
$$

\subsubsection{Παρεμβολή Lagrange (Lagrange Interpolation)}

Πρόκειται από τις πιο γνωστές και βασικές μεθόδους παρεμβολής. 
\begin{definition}
\textbf{Παρεμβολή Lagrange :}
Έστω ότι θέλουμε να παρεμβάλουμε $k+1$ σημεία για τα οποία μας δίνονται οι τιμές τους στη μορφή $\left(x_{i}, y_{i}\right)$ όπου $i \in 0, 1, \dots k$ και $x_i \ne x_j \forall i, j$. Τότε αρκεί να παρατηρήσουμε ότι χρειαζόμαστε μια συνάρτηση $l_j$ με την παρακάτω μορφή :
$$
l_{j}(x) =\begin{cases}1, x=x_j\\ 0, x = x_i\\ \text{κάτι άλλο}, \text{σε οποιαδήποτε άλλη περίπτωση}\end{cases}
$$
Τότε μπορούμε να δημιουργήσουμε μια παρεμβολή από τα σημεία που μας δίνονται ως εξής :
$$
L(x):=\sum_{j=0}^{k} y_{j} l_{j}(x)
$$
Η $L(x)$ μπορούμε να επαληθέυσουμε ότι παρεμβάλει τα σημεία $(x_i, y_i)$ αφού $L(x_j)=y_j \forall j \in 0, 1, \dots, k$
Τέλος αρκεί να βρούμε μια κατάλληλη συνάρτηση $l_j(x)$. Αυτή προφανώς μπορεί να οριστεί ως εξής :
$$
\ell_{j}(x):=\prod_{\substack{0 \leq m \leq k \\ m \neq j}} \frac{x-x_{m}}{x_{j}-x_{m}}=\frac{\left(x-x_{0}\right)}{\left(x_{j}-x_{0}\right)} \cdots \frac{\left(x-x_{j-1}\right)}{\left(x_{j}-x_{j-1}\right)} \frac{\left(x-x_{j+1}\right)}{\left(x_{j}-x_{j+1}\right)} \cdots \frac{\left(x-x_{k}\right)}{\left(x_{j}-x_{k}\right)}
$$
Μπορούμε να επαληθεύσουμε ότι η $l_j(x)$ όντως έχει τα χαρακτηριστικά που επιθυμούμε : 
\begin{itemize}
    \item Για $j=i$ : 
    $$
        \ell_{j}\left(x_{j}\right):=\prod_{m \neq j} \frac{x_{j}-x_{m}}{x_{j}-x_{m}}=1
    $$
    \item Για $j \ne i$
    $$
        \ell_{j}\left(x_{i}\right)=\prod_{m \neq j} \frac{x_{i}-x_{m}}{x_{j}-x_{m}}=\frac{\left(x_{i}-x_{0}\right)}{\left(x_{j}-x_{0}\right)} \cdots     \frac{\left(x_{i}-x_{i}\right)}{\left(x_{j}-x_{i}\right)} \cdots \frac{\left(x_{i}-x_{k}\right)}{\left(x_{j}-x_{k}\right)}=0 .
    $$
    \item Για οποιαδήποτε άλλη περίπτωση
    $$
        \ell_{j}\left(x_{j}\right):=\prod_{m \neq j} \frac{x_{j}-x_{m}}{x_{j}-x_{m}}
    $$
\end{itemize}

\end{definition}