\section{Αλγοριθμικό Υπόβαθρο}
\label{section:algorithmic-background}

Στην ενότητα αυτή παραθέτουμε και αναλύουμε, χωρίς εμβάθυνση, ορισμένους βασικούς ορισμούς από τη Θεωρία Πολυπλοκότητας. Κύρια βιβλιογραφική πηγή στη συγκεκριμένη ενότητα αποτέλεσε η \cite{Bauer2011}.

\subsection{Βασικοί Ορισμοί}

Αρχικά, θεωρούμε σημαντικό να διασαφηνίσουμε τον όρο αποδοτικός αλγόριθμος στα πλαίσια της κρυπτογραφίας. Αυτός ορίζεται ως εξής :

\begin{definition}
\textbf{Αποτελεσματικός αλγόριθμος (efficient algorithm)} : Κάθε αλγόριθμος ο οποίος έχει χρονική και χωρική πολυπλοκότητα $O(poly(n))$, όπου $n$ είναι το μέγεθος της εισόδου του αλγορίθμου.
\end{definition}

Ακόμα, θέλουμε να διασαφηνίσουμε τον όρο Boolean δίκτυο στα πλαίσια του SMPC, αυτό θα μας είναι χρήσιμο στο Κεφάλαιο \ref{chapter:SMPC}, σε SMPC πρωτόκολλα που πραγματοποιούν υπολογισμούς πάνω σε Boolean δίκτυα. Ένα Boolean δίκτυο ορίζεται ως εξής :

\begin{definition}
\textbf{Boolean Δίκτυο} : Ένα κατευθυνόμενο ακυκλικό γράφημα (Directed Acyclic Graph ή DAG), $G=(E, V)$, όπου κάθε κόμβος $v \in V$ είναι μια Boolean πύλη, και κάθε ακμή μια σύνδεση μεταξύ των boolean πυλών. Για κάθε κόμβο $v \in V$, $deg^-(v) \ge 1$ και $deg^+(v) = 1$.
\end{definition}

Παραθέτουμε επίσης τον ορισμό του Μαντείου, που απαντάται στο Κεφάλαιο \ref{chapter:security}, υπό την μορφή του Τυχαίου Μαντείου που είναι και η πιο συνήθης μορφή του που τον συναντάμε στον κλάδο της κρυπτογραφίας.

\begin{definition}
\textbf{Μαντείο (Oracle)} : Πρόκειται για έναν αλγόριθμο μαύρου κουτιού o οποίος δέχεται ερωτήσεις και δίνει απαντήσεις.
\end{definition}

\subsection{Διαδραστικές Αποδείξεις}

Τέλος, θα κάνουμε μια μικρή αναφορά στις Διαδραστικές Αποδείξεις στις οποίες βασίζονται οι Διαδραστικές ή Μη Αποδείξεις Μηδενικής Γνώσης, οι οποίες αποτελούν βασικό εργαλείο της σύγχρονης κρυπτογραφίας και στις οποίες αναφερθήκαμε στο Κεφάλαιο \ref{chapter:cryptography}.

Για να μπορέσουμε να αντιληφθούμε την έννοια και την χρησιμότητα των αποδείξεων πρέπει να ξεκινήσουμε τη μελέτη από το παραδοσιακό σύστημα αποδείξεων.  Ας πάρουμε για παράδειγμα μια απόδειξη για μια Boolean έκφραση $x$ που θέλουμε να αποδείξουμε ότι είναι, μη ικανοποιήσιμη, δηλαδή ότι δεν υπάρχει ανάθεση τιμών που να οδηγεί σε αληθή αποτίμηση. Ονομάζουμε \textbf{Γλώσσα $L$}, το (άπειρο σύνολο) όλων των Boolean εκφράσεων που είναι μη ικανοποιήσιμες και ονομάζουμε \textbf{Μάρτυρα ή Πιστοποιητικό} ενός στοιχείου $x$ την απόδειξη ότι $x \in L$. Το συγκεκριμένο πρόβλημα ανάγεται στο να αποδείξουμε η έκφραση $x$ ανήκει στη γλώσσα $\UNSAT$, δηλαδή $x \in \UNSAT$.
Επίσης, γνωρίζουμε ότι $\UNSAT \in \coNP$. Επιστρέφοντας στο παράδειγμα μας, ας υποθέσουμε ότι ένας άνθρωπος, o \textbf{Αποδεικνύον (Prover) $P$} (θεωρούμε ότι διαθέτει υπολογιστικά απεριόριστους πόρους), θέλει να αποδείξει σε κάποιον άλλο, τον \textbf{Επαληθευτή (Verifier) $V$} (θεωρούμε ότι διαθέτει υπολογιστικά περιορισμένους πόρους), ότι $x \in \UNSAT$. Σε μια μη διαδραστική απόδειξη για το $\UNSAT$, η καλύτερη λύση που γνωρίζουμε είναι ο $P$ να γράψει τις αποτιμήσεις της $x$ για όλες τις πιθανές τιμές των μεταβλητών της σε ένα χαρτί\footnote{Η κλάση $\coNP$ περιέχει όλα τα προβλήματα για τα οποία εικάζουμε ότι δεν μπορεί να υπάρξει πολυωνυμική στο μέγεθος της εισόδου απόδειξη.}, ο $V$ να πάρει το χαρτί να το διαβάσει και αν πειστεί από την απόδειξη να δηλώσει ότι αποδέχεται ότι $x \in \UNSAT$. Στο παράδειγμα μας, αν θεωρήσουμε ότι υπάρχουν $n$ μεταβλητές στην έκφραση $x$ τότε το μέγεθος της απόδειξης είναι τουλάχιστον $Ω(2^n)$. Αυτό σημαίνει ότι ο $V$, δεν μπορεί να διαθέτει υπολογιστικά περιορισμένους πόρους παρά μόνο πόρους εκθετικούς στο μέγεθος του $n$ για να μπορέσει να διαβάσει όλη την απόδειξη από το χαρτί. Παρατηρούμε ότι αυτό δε θα ήταν το ίδιο με ένα πρόβλημα $x \in \NP$, αφού ο μάρτυρας στη συγκεκριμένη περίπτωση είναι πολυωνυμικός στο μέγεθος της εισόδου και έτσι ένας $V$ με υπολογιστικά περιορισμένους πόρους είναι αρκετός. Στην περίπτωση μάλιστα του $\SAT$ το μέγεθος του μάρτυρα είναι $O(n)$, δηλαδή γραμμικό στο μέγεθος της εισόδου. Οι Διαδραστικές Αποδείξεις (Interactive Proofs ή $\IP$) ουσιαστικά είναι μια μέθοδος αποδείξεων, μια αφηρημένη υπολογιστική μηχανή, η οποία μας επιτρέπει να διατηρήσουμε την ισχύ του $P$ άπειρη και την ισχύ του $V$ υπολογιστικά περιορισμένη, που συνεπάγεται ότι και το μέγεθος του μάρτυρα θα πρέπει να είναι πολυωνυμικό, με το μοναδικό μειονέκτημα να υπάρχει αυθαίρετη πιθανότητα, τυπικά την ορίζουμε ως $\le \sfrac{1}{3}$, με την οποία ο $P$ να μπορεί να πείσει τον $V$ να αποδεχθεί χωρίς ο ίδιος ο $P$ να γνωρίζει πραγματικά έναν μάρτυρα. Δηλαδή το μοναδικό μειονέκτημα των $\IP$ είναι ότι μπορεί να υπάρχουν πλαστοί μάρτυρες/πιστοποιητικά τα οποία να μπορούν να οδηγήσουν τον $V$ σε πλάνη. Οι διαδραστικές αποδείξεις παρουσιάστηκαν για πρώτη φορά στην βιβλιογραφία από τον Babay στην εργασία \cite{10.1145/22145.22192} \improvement{Investigate further}. Μπορούμε τώρα να ορίσουμε με πιο αυστηρό τρόπο όλα τα παραπάνω.

\begin{definition}
\textbf{Διάδραση μεταξύ ντετερμινιστικών συναρτήσεων} : Έστω συναρτήσεις $f, g$ : $\{0,1\}^{*} \rightarrow\{0,1\}^{*}$. Μια $k$-γύρων διάδραση μεταξύ των $f$ και $g$ σε είσοδο $x \in\{0,1\}^{*}$, την οποία συμβολίζουμε ως $\langle f, g\rangle(x)$, είναι η ακολουθία των συμβολοσειρών $a_{1}, \ldots, a_{k} \in\{0,1\}^{*}$ που ορίζονται ως εξής :

\[
\begin{aligned}
a_{1} &=f(x) \\
a_{2} &=g(x, a_{1}) \\
\ldots & \\
a_{2 i+1} &=f(x, a_{1}, \ldots, a_{2 i}) \\
a_{2 i+2} &=g(x, a_{1}, \ldots, a_{2 i+1})
\end{aligned}
\]

όπου $x$ είναι η είσοδος που θέλουμε να αποδείξουμε ότι $x \in L$
\end{definition}

Στην περίπτωση της γλώσσας $\UNSAT$, το $x$ είναι η Boolean έκφραση που θέλουμε να αποδείξουμε ότι $x \in \UNSAT$. Προφανώς, αν ο οι συναρτήσεις $f$ και $g$ έχουν εσωτερική κατάσταση που μπορεί να διατηρήσει το ιστορικών των προηγούμενων μηνυμάτων, δεν χρειάζεται όλο το ιστορικό μηνυμάτων να αποστέλλεται σε κάθε αλληλεπίδραση. Τώρα μπορούμε να ορίσουμε αυστηρά την κλάση των Ντετερμινιστικών Διαδραστικών Αποδείξεων :

\begin{definition}
\textbf{Κλάση Ντετερμινιστικών Διαδραστικών Συναρτήσεων (Deterministic Interactive Proofs ή $\dIP$)}: Έστω $P$ μια ντετερμινιστική ΤΜ που είναι απεριόριστης ισχύος. Η γλώσσα $L \in \text{dIP}$ λέμε ότι διαθέτει ένα ντετερμινιστικό σύστημα διαδραστικής απόδειξης αν υπάρχει ντετερμινιστική TM $V$ με είσοδο $x, a_{1}, \ldots, a_{i}$ και χρονική πολυπλοκότητα $O(poly(|x|))$ που ικανοποιεί τις παρακάτω ιδιότητες :
\begin{itemize}
    \item \textbf{Πληρότητα (Completeness)} : $x \in L \Rightarrow \exists P:\{0,1\}^{*} \rightarrow\{0,1\}^{*} \wedge \text { out }_{V}\langle V, P\rangle(x) = 1$
    \item \textbf{Ορθότητα (Soundness)} : $x \notin L \Rightarrow \forall P:\{0,1\}^{*} \rightarrow\{0,1\}^{*} \text { out }_{V}\langle V, P\rangle(x) = 0$
\end{itemize}
\end{definition}

Μερικές σημείωσεις σχετικά με τον παραπάνω ορισμό είναι ότι αφού ο $V$ έχει χρονική πολυπλοκότητα $O(poly(|x|))$, προφανώς και μπορεί να επεξεργαστεί $O(poly(|x|))$ αριθμό μηνυμάτων και άρα το μέγεθος του μάρτυρα $|w|=O(poly(|x|))$. Επίσης, ρόλος του $V$ στον ορισμό αυτό είναι διπλός, είναι η ΤΜ που αλληλεπιδρά με τον $P$ για την παραγωγή των μηνυμάτων της διάδρασης, δηλαδή του μάρτυρα, αλλά και η ΤΜ που θα επαληθεύσει τον μάρτυρα. Σχετικά με την κλάση  $\dIP$ μπορεί εύκολα να αποδειχθεί ότι $\dIP = \NP$. Ωστόσο, μεγαλύτερο ενδιαφέρον προκύπτει αν θεωρήσουμε μη ντετερμινιστικό τον Επαληθευτή $V$. Τότε η κλάση που προκύπτει είναι η γνωστή $\IP$ και ορίζεται ως εξής :

\begin{definition}
\textbf{Κλάση (Μη Ντετερμινιστικών) Διαδραστικών Αποδείξεων ((Non-Deter\-ministic) \EN{Interactive Proofs} ή $\IP$)} : Έστω $P$ μια ντετερμινιστική ΤΜ που είναι απεριόριστης ισχύος. Η γλώσσα $L$ λέμε ότι ανήκει στην κλάση $\IP$, δηλαδή $L \in IP$ αν υπάρχει μη ντετερμινιστική ΤΜ $V$ με $x, r, a_{1}, \ldots, a_{i}, V$ και χρονική πολυπλοκότητα σε $O(poly(|x|))$ που ικανοποιεί τις παρακάτω ιδιότητες :
\begin{itemize}
    \item \textbf{Πληρότητα (Completeness)} : $x \in L \Rightarrow \exists \operatorname{Pr}\left[\text {out}_{V}\langle V, P\rangle(x)=1\right] \geq (ε=) \dfrac{2}{3}$
    \item \textbf{Ορθότητα (Soundness)} : $x \notin L \Rightarrow \forall \operatorname{Pr}\left[\text {out}_{V}\langle V, P\rangle(x)=1\right] \leq 1 - ε$
\end{itemize}
\end{definition}

Σχετικά με τον παραπάνω ορισμό, γνωρίζουμε ότι η έξοδος μιας μη ντετερμινιστικής ΤΜ είναι πιθανοτική και μπορούμε να σκεφτούμε το σύνολο των εισόδων της ως μια πιθανοτική κατανομή. Σχετικά με τη σταθερά $ε$ έχει αποδειχθεί ότι μπορούμε να έχουμε Τέλεια Πληρότητα και Τέλεια Ορθότητα μόνο για γλώσσες στην κλάση $\NP$. Οι τιμή $\sfrac{2}{3}$ είναι αυθαίρετη και αν επαναλάβουμε το πρωτόκολλο $m$ αριθμό φορών μπορούμε να την αντικαταστήσουμε με $1-negl(m)$. Έτσι ουσιαστικά με τους ορισμούς του Κεφαλαίο \ref{chapter:security} μπορούμε να πούμε ότι έχουμε \textbf{Υπολογιστική Πληρότητα (Computational Completeness)} και \textbf{Υπολογιστική Ορθότητα (Computational Soundness)}. Ένα από τα σημαντικότερα ευρήματα του κλάδου είναι ότι $IP=PSPACE$. Η απόδειξη παρουσιάστηκε από τον Shamir το 1992 στην \cite{shamir1992ip}. Τέλος, δεν μπορούμε να παραλείψουμε ένα από τα πρώτα IP πρωτόκολλα, το Sumcheck, που προτάθηκαν στην \cite{lund1992algebraic} για τη διαδραστική απόδειξη της $\UNSAT$ γλώσσας.