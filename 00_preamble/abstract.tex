\chapter*{Περίληψη}
\addcontentsline{toc}{chapter}{Περίληψη}

\pagestyle{plain}

Στην παρούσα διπλωματική εργασία γίνεται μια εισαγωγή και μελέτη σε διαφορετικά πρωτόκολλα του Υπολογισμού Πολλών Μερών (Secure Multi Party Computation ή SMPC). Πριν ξεκινήσουμε τη μελέτη αυτή, γίνεται μια εισαγωγή στην κρυπτογραφία και στη θεωρητική αποδείξιμη ασφάλεια παρουσιάζοντας όλο το απαραίτητο θεωρητικό υπόβαθρο, μαθηματικό και αλγοριθμικό, που απαιτείται για την βαθύτερη κατανόηση του αντικειμένου.. Στην συνέχεια υλοποιούμε ένα μέρος μιας BLAS βιβλιοθήκης  Στην εισαγωγή αυτή εστιάζουμε κυρίως σε νέες πρακτικές και εργαλεία που χρησιμοποιούνται σήμερα

\vspace{0.3cm}

\begin{flushleft}
\huge\textbf{Abstract}
\end{flushleft}

On the current diploma thesis we make an introduction and study on various modern protocols of Secure Multi Party Computation (SMPC). Before we begin the our study on SMPC, we make an introduction to cryptography and to theoretically provable security, presenting all the required mathematical and algorithmic background required to deeply understand the topic.