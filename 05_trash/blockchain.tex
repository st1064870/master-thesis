\chapter{Blockchain}

\section{Blockchain}

\subsection{Ιστορική αναδρομή}

Ο όρος Blockchain εισηχθεί για πρώτη φορά στην βιβλιογραφία από τον Satoshi Nakamototo στο Bitcoin Whitepaper \improvement{reference}, το 2008, και υλοποιήθηκε το 2009 από τον ίδιο για να βασιστεί σε αυτό το γνωστό πλέον κρυπτονόμισμα Bitcoin. Στη πρώτη του μορφή, το Blockchain αποτέλεσε μια αποκεντρωμένη δομή δεδομένων χρονουπογραφών η οποία χρησιμοποιεί το οικονομικό κίνητρο ώστε να αποτρέψει την αντιστρεψιμότητα και την πλαστογράφηση αυτών. Πρόκειται ουσιαστικά για ένα Κατανεμημένο Κατάστιχο Συναλλαγών (Distributed Transactions Ledger). Το Blockchain, όπως παρουσιάστηκε από τον Nakamoto, δεν βασίστηκε σε καινοτόμες για την εποχή ιδέες όμως ο συνδυασμός τους αποτέλεσε μια μεγάλη καινοτομία. Βασίστηκε στην εργασία των Stokk and Sornetta πάνω στις Ψηφιακές Χρονουπογραφές (Digital timestamping) \improvement{referece}, τον αλγόριτημο hashcash \improvement{reference} που είχε προταθεί για την αντιμετώπιση της αλόγιστης χρήσης πόρων (π.χ. spam mails), τις Κρυπτογραφικές Ψηφιακές Υπογραφές \improvement{reference} και τα Δέντρα Merkle (Merkle Trees) που έχουν προταθεί για γρήγορη επαλήθευση της ακεραιότητας δεδομένων σε δικτυακά πρωτόκολλα.

\subsection{Σύγχρονες τάσεις στο Blockchain}

To Blockchain τα τελευταία χρόνια έχει απομακρυνθεί από την στενή του σχέση με τα κρυπτονομίσματα που είχε όταν πρώτο εισήχθει στην βιβλιογραφία και πλέον αποτελεί αυτούσιο μια δομή δεδομένων που γίνεται ερεύνα γύρω από αυτήν αλλά και έχει διαπιστωθεί ότι μπορεί να λύσει και να παρακάμψει πλήθος προβλημάτων στον επιχειρηματικό και δημόσιο τομέα. Όπως για παράδειγμα προβλήματα που σχετίζονται με την εφοδιαστική αλυσίδα \improvement{reference}, τον τραπεζικό τομέα \improvement{reference}, τις ψηφιακές συναλλαγές \improvement{reference}, τα ψηφιακά τεκμήρια ιδιοκτησίας \improvement{reference} και την διαφανή διαχείριση πόρων \improvement{reference}, προβλήματα που προέρχονται από το Βιομηχανικό Διαδίκτυο των Πραγμάτων (Industrial Internet of Things ή IIoT) κ.α. Ωστόσο, πρέπει να αναφερθεί ότι αυτή η πληθώρα προβλημάτων συνήθως δεν επιλύονται από την αρχική διατύπωση του Blockchain ως ένα Κατανεμημένο Κατάστιχο Συναλλαγών αλλά από βελτιώσεις αυτής της διατύπωσης που συμπεριλαμβάνουν χαρακτηριστικά όπως αυτά της εκτέλεση Έξυπνων Συμβολαίων (Smart Contracts) \improvement{reference}, των Τυφλών Υπογραφών (Blind Signatures) \improvement{reference}, των Αποδείξεων Μηδενικής Γνώσης (Zero-Knowledge Proofs) \improvement{reference}. Έτσι, ουσιαστικά πλέον το Blockchain έχει εδραιωθεί στην βιβλιογραφία ως μια πολύ εύπλαστη δομή δεδομένων στην οποία προσθέτονται ad-hoc χαρακτηριστικά και λειτουργίες (είτε οριζόντια είτε κάθετα) για την επίτευξη του επιθυμητού αποτελέσματος. Σε αυτό το σημείο πρέπει να αναφερθεί τα Blockchain χωρίζονται σε δύο κατηγορίες ως προς το ποιος έχει πρόσβαση στο αποκεντρωμένο δίκτυο που το συνηστά, τα δημόσια Blockchain και τα ιδιωτικά.