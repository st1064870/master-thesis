\chapter{Επίλογος}
\label{chapter:postamble}

Σε αυτό το σημείο, γίνεται μία σύντομη σύνοψη των προηγούμενων κεφαλαίων. Δίνεται περισσότερη έμφαση κυρίως στο κεφαλαίο της υλοποίησης, που είναι και ουσιαστικά η κύρια πρόταση αυτής της εργασίας, πέραν της μελέτης των SMPC πρωτοκόλλων και τις ανάλυσης όλων των προαπαιτούμενων γνώσεων για την κατανόηση αυτών.

\section{Αποτελέσματα και Συμπεράσματα}

Στο θεωρητικό κομμάτι αυτής της εργασίας έγινε μια εισαγωγή στην κρυπτογραφία, στη θεωρητική ασφάλεια καθώς και σε μεθόδους απόδειξης της, με απότερο σκοπό την ανάλυση πρωτοκόλλων Ασφαλού Υπολογισμού Πολλών Μερών (SMPC). Στα πλαίσια της κρυπτογραφίας αναλύθηκαν σύγχρονα εργαλεία όπως Σχήματα Διαμοίρασης Μυστικών, Σχήματα Δέσμευσης, Σχήματα Ανυποψίαστης και Αποδείξεις Μηδενικής Γνώστης. Παράλληλα με αυτά αναλύθηκε όλο το μαθηματικό υπόβαθρο που απαιτείται για τη βαθύτερη κατανόηση τους. Σχετικά με τη θεωρητική ασφάλεια, αναλύθηκαν οι βασικοί ορισμοί και οι βασικές μεθόδοι απόδειξης της, η Μέθοδος Μεταπήδησης μεταξύ Παιχνιδιών και η Μέθοδος Προσομοίωσης. Αφού καλύφθηκε όλο το απαραίτητο υπόβαθρο, αναλύθηκαν τέσσερα πρωτόκολλα SMPC, το Πρωτόκολλο Μπερδεμένων Δικτύων Yao, το BMR, το GMW και το SPDZ. Τα πρώτα τρία από αυτά αποτελούν κλασσικά πρωτόκολλα τα οποία ήταν υποψήφια για να χρησιμοποιηθούν στην υλοποίηση, ενώ το τελευταίο είναι από τα πιο σύγχρονα και επιτυχημένα πρωτόκολλα το οποίο είναι ασφαλές ενάντια σε ενεργητικούς αντιπάλους.

Στο πρακτικό κομμάτι αυτής της εργασίας υλοποιήθηκε, και περιγράφηκε στο Κεφάλαιο \ref{chapter:implementation}, η βιβλιοθήκη MPC-BLAS, μια 2MPC BLAS Level-1 βιβλιοθήκη για πραγματικούς αριθμούς κινητής υποδιαστολής μονής ακρίβειας για δύο συμμετέχοντες. Η βιβλιοθήκη αυτή είναι η πρώτη βιβλιοθήκη αυτού του είδους, που προτείνεται στη βιβλιογραφία, τη στιγμή που γράφεται αυτή η εργασία.

Στα πλαίσια της υλοποίησης εκτελέστηκαν επίσης δύο πειράματα. Προσπαθήσαμε να συγκρίνουμε το ποσοστό της επιβάρυνσης που εισάγει μια SMPC υλοποίηση για τους αλγορίθμους που υλοποιήσαμε. Παρατηρήσαμε ότι η επιβάρυνση που εισάγεται είναι γραμμική και είναι της τάξης των 10.000-100.000 χρονικών μονάδων. Δηλαδή, αν μια εκτέλεση κάποιας συνάρτησης μια βιβλιοθήκης CBLAS απαιτούσε 1 δευτερόλεπτο, η αντίστοιχη κλήση στην περίπτωση της MPC-BLAS θα απαιτούσε 10.000-100.000 δευτερόλεπτα. Αυτό ίσως ακούγεται ιδιαίτερα αποθαρρυντικό για μια τυπική/εμπορική χρήση αυτής της βιβλιοθήκης. Ωστόσο, εκτιμούμε πως υπάρχει τεράστιο περιθώριο βελτίωσης, ίσως και κατά αρκετές τάξεις μεγέθους και αυτό αποδεικνύεται από το γεγονός πως τα πρωτόκολλα SMPC έχουν σήμερα παραδείγματα εμπορικών εφαρμογών και μάλιστα για μεγάλους όγκους δεδομένων. Να σημειώσουμε βέβαια πως περισσότερα από τα παραδείγματα αυτά επικεντρώνονται κυρίως σε σχετικά μικρό αριθμό συμμετεχόντων, συνήθως μικρότερο των 10. Πολλές ιδέες για τη βελτίωση της απόδοσης της βιβλιοθήκης αναλύονται στην επόμενη ενότητα. Τέλος, θεωρούμε πως μια βελτιωμένη, ως προς την ταχύτητα, μορφή MPC-BLAS μπορεί να αποκτήσει πρακτική εφαρμογή. Ο ισχυρισμός αυτός βασίζεται στο γεγονός ότι υπάρχουν πάρα πολλά προγράμματα και βιβλιοθήκες που χρησιμοποιούν BLAS πράξεις, όπως το MATLAB και η NumPy στις οποίες βασίζονται εφαρμογές που θα επιθυμούσαν την ιδιότητα της ιδιωτικότητας.

\section{Ιδέες και θέματα μελλοντικής μελέτης}

Στην Ενότητα αυτή θα παρουσιάσουμε ορισμένες ιδέες για βελτίωση της βιβλιοθήκης MPC-BLAS που υλοποιήσαμε. Στις περισσότερες περιπτώσεις η βιβλιοθήκη ABY στην οποία βασίζεται η MPC-BLAS αποτελεί τον περιοριστικό παράγοντα και άρα μια βελτίωση ή προσθήκη στη βιβλιοθήκη ABY συνεπάγεται με μικρές αλλαγές βελτίωση της MPC-BLAS.

\subsection{Υλοποίηση των BLAS-2 και BLAS-3 υπορουτινών και ολοκλήρωση μιας εύχρηστης βιβλιοθήκης}
Η υλοποίηση των λειτουργιών BLAS-2 και BLAS-3 ήταν εξ' αρχής εκτός των σκοπών αυτής της εργασίας, αφού κύριο μέλημα της ήταν η μελέτη του κλάδου του SMPC ξεκινώντας από τα πρώτα του βήματα και φτάνοντας ως τη σύγχρονη βιβλιογραφία, τόσο τη θεωρητική όσο και την υλοποιητική και τέλος η προσπάθεια πειραματισμού και εφαρμογής αυτών σε μια καινοτόμα υλοποίηση. Η υλοποίηση σε καμία περίπτωση δεν είναι πλήρης και δεν υπήρχε ο χρόνος για να συμβεί αυτό. Έτσι, η υλοποίηση των BLAS-2 και BLAS-3 υπορουτινών της βιβλιοθήκης είναι το σημαντικότερο κομμάτι του παζλ που λείπει αυτή τη στιγμή από την υλοποίηση για να ολοκληρωθεί και να εξασφαλίσει έτσι πιθανότητες να χρησιμοποιηθεί και σε άλλα πειράματα στη βιβλιογραφία. Σαν επόμενο βήμα από αυτό θα ήταν η δημιουργία διεπαφής/δεσιμάτων (bindings) και σε άλλες γλώσσες προγραμματισμού ή βιβλιοθήκες προκειμένου να γίνει πιο εύχρηστη. Ας μην ξεχνάμε άλλωστε, ότι τα υλοποιητικά παράγωγα της κρυπτογραφίας, στην οποιαδήποτε μορφή τους, καλούνται να παίξουν τον ρόλο του διαφανούς ενδιάμεσου σε έναν χρήστη ή έναν προγραμματιστή, ρόλος που πιστεύουμε ότι ως επί το πλείστον αποτυγχάνουν σε σημαντικό βαθμό να παίξουν και δεν είναι καθόλου εύκολο να επιτύχεις.

\subsection{Υλοποίηση αποδοτικότερου αλγορίθμου Κινητής Υποδιαστολής}
Όπως προαναφέραμε η βιβλιοθήκη ABY υποστηρίζει πράξεις κινητής υποδιαστολής μόνο μέσω Boolean κυκλώματος, το οποίο είναι μια υλοποίηση μιας Μονάδας Κινητής Υποδιαστολής σε γλώσσα περιγραφής υλικού (π.χ. Verilog, VHDL κτλ.) η οποία στη συνέχεια έχει μεταφραστεί και μετατραπεί σε κατάλληλη μορφή ώστε να μπορεί να την διαβάσει το ABY με σκοπό να δημιουργήσει το αντίστοιχο Μπερδεμένο Δίκτυο Yao. Είναι προφανές ότι παρά τις τεχνικές επιτάχυνσης των δικτύων αυτών, η πολυπλοκότητα επικοινωνίας τους δεν μπορεί να συγκριθεί με αυτή αντίστοιχων ομομορφικών πρωτοκόλλων που υποστηρίζουν αυτές τις πράξεις \cite{cryptoeprint:2022/322} \cite{alxd1}. Αυτό κυρίως οφείλεται στο ότι η πολυπλοκότητα επικοινωνίας τους εξαρτάται από το βάθος του κυκλώματος που χρησιμοποιούν για να αναπαραστήσουν τον υπολογισμό, όπου στην παρούσα περίπτωση είναι αρκετά μεγάλο. Είναι σημαντικό να αναφέρουμε, ότι υποστήριξη πράξεων κινητής υποδιαστολής που συμμορφώνονται σε συγκεκριμένα πρότυπα, όπως για παράδειγμα το IEEE754 \cite{8766229}, δεν είναι μια ιδιαίτερα απλή αλγοριθμική και υλοποιητική διαδικασία \cite{10.5555/1096483} η οποία γίνεται ακόμα δυσκολότερη όταν θέλουμε να υποστηρίξουμε και συναρτήσεις όπως η εύρεση ρίζας ($SQRT$, που χρειάζεται για παράδειγμα στον υπολογισμό του μέτρου ενός διανύσματος στην περίπτωση του BLAS, ή το ημίτονο ($SIN$) που χρησιμοποιείται σε έναν πίνακα περιστροφής. Η διαδικασία γίνεται ακόμα πιο δυσμενής αν αναλογιστούμε τους παράγοντες της ασφάλειας και της αποδοτικότητας που εισάγει ο MPC. Μόλις το 2021 αναπτύχθηκε η πρώτη σουίτα αλγορίθμων που υποστηρίζουν πράξεις κινητής υποδιαστολής καθώς και ένα πλήθος από συναρτήσεις όπως αυτή του ημίτονου και της ρίζας που αναφέραμε, στην εργασία \cite{cryptoeprint:2022/322} με όνομα SECFLOAT. Η SECFLOAT βασίζεται στον διαμοιρασμό μυστικών και υποστηρίζει πράξεις κινητής υποδιαστολής μέχρι 32-bit με συμμόρφωση στο πρότυπο IEE754. Η υλοποίηση αυτής της σουίτας αλγορίθμων θα αύξανε δραματικά την απόδοση του συστήματος μας αφού οι BLAS συναρτήσεις χρησιμοποιούν αποκλειστικά πράξεις κινητής υποδιαστολής. Πιο συγκεκριμένα, οι συγγραφείς της SECFLOAT την\ σύγκριναν με την απόδοση του ABY σε πράξεις κινητής υποδιαστολής και παρατήρησαν ότι είναι 5-11 φορές γρηγορότερη ανάλογα με την πράξη που επιλέγεται. Ο μοναδικός περιοριστικός παράγοντάς είναι ότι η SECFLOAT έτσι όπως έχει προταθεί στη βιβλιογραφία υποστηρίζει πράξεις μόνο μέχρι 32-bit κάτι που θα επέτρεπε μόνο περίπου το 1/4 μιας βιβλιοθήκης BLAS να βασιστεί σε αυτήν καθώς στα περισσότερα συστήματα τα 32-bit αντιστοιχούν στον τύπου float. Εκτιμούμε όμως ότι οι αλγόριθμοι της SECFLOAT με μικρές αλλαγές μπορούν να υποστηρίξουν πράξεις κινητής υποδιαστολής μέχρι και 64-bit.

\subsection{Επέκταση της ABY για SMPC ή χρήση βιβλιοθήκης για υποστήριξη SMPC}
Η βιβλιοθήκη ABY υποστηρίζει μόνο υπολογισμού για πλαίσια του 2PC. Αυτό ίσως αποτελεί έναν περιοριστικό παράγοντα για τη βιβλιοθήκη μας. Θα μπορούσαμε να εξετάσουμε να την βασίσουμε σε κάποια άλλη βιβλιοθήκη που να υποστηρίζει εξ'αρχής SMPC πρωτόκολλα. Όπως αναφέρουμε και στο Κεφάλαιο \ref{chapter:SMPC}, πριν την υλοποίηση η προσπάθεια μας για εύρεση κατάλληλης βιβλιοθήκης για SMPC απέβη άκαρπη. Στην υλοποίηση αυτής της ιδέας, παρότι θεωρούμε ότι θα είχε πολύ θετικό αντίκτυπο, κρίνουμε πως είναι ιδιαίτερα δύσκολη διότι είτε θα πρέπει να επεκτείνουμε τη βιβλιοθήκη ABY για υποστήριξη SMPC είτε να χρησιμοποιήσουμε κάποια δύσχρηστη βιβλιοθήκη που να υποστηρίζει υπολογισμούς σε αυτό το πλαίσιο όπως η MP-SPDZ \cite{mp-spdz}.

\subsection{Επέκταση της ABY ή χρήση βιβλιοθήκης με υποστήριξη πρωτοκόλλων ασφαλών ενάντια σε Ενεργητικούς Αντιπάλους}
Η βιβλιοθήκη ABY δεν υποστηρίζει πρωτόκολλα ασφαλή απέναντι σε ενεργητικούς αντιπάλους, παράπονο ενάντια σε παθητικούς αντιπάλους και άρα η παρούσα έκδοση της MPC-BLAS δεν υποστηρίζει αυτού του είδους πρωτόκολλα. Στο Κεφάλαιο \ref{chapter:SMPC} μελετήθηκε το πρωτόκολλο SPDZ το οποίο, ως και σήμερα, με παραλλαγές του, αποτελεί από τα πιο αποδοτικά πρωτόκολλα ασφαλή ενάντια σε ενεργητικούς αντιπάλους που έχουν προταθεί στη βιβλιογραφία. Θα μπορούσαμε να επεκτείνουμε τη βιβλιοθήκη ABY στο να υλοποιεί το SDPZ και στη συνέχεια να το χρησιμοποιήσουμε στη βιβλιοθήκη MPCBLAS. Η συγκεκριμένη πρόταση γίνεται διότι, παρότι υπάρχουν ορισμένες υλοποιήσεις το SPDZ όπως οι \cite{mp-spdz,aly2021scale,FRESCO}, πολλές από αυτές δε διαθέτουν ιδιαίτερα εύχρηστη διεπαφή για C/C++, όπως η \cite{aly2021scale}, δεν διαθέτουν κατάλληλη τεκμηρίωση της διεπαφής, όπως η \cite{mp-spdz}, ή δε διαθέτουν διεπαφή για C/C++, όπως η \cite{FRESCO}. Ωστόσο, στην περίπτωση που θέλουμε να επιτύχουμε και υψηλή απόδοση στη βιβλιοθήκη θα πρέπει να διερευνηθεί το κατά πόσο μπορεί η σουίτα αλγορίθμων SECFLOAT να υλοποιηθεί με βάση το SPDZ.